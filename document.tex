\documentclass[10pt,twocolumn]{article}

% use the oxycomps style file
\usepackage{oxycomps}

% usage: \fixme[comments describing issue]{text to be fixed}
% define \fixme as not doing anything special
\newcommand{\fixme}[2][]{#2}
% overwrite it so it shows up as red
\renewcommand{\fixme}[2][]{\textcolor{red}{#2}}
% overwrite it again so related text shows as footnotes
%\renewcommand{\fixme}[2][]{\textcolor{red}{#2\footnote{#1}}}

% read references.bib for the bibtex data
\bibliography{references}

% include metadata in the generated pdf file
\pdfinfo{
    /Title (Git and LaTeX Worksheet)
    /Author (Justin Li)
}

% set the title and author information
\title{Git and \LaTeX Worksheet}
\author{Justin Li}
\affiliation{Occidental College}
\email{justinnhli@oxy.edu}

\begin{document}

\maketitle

\section{Instructions}

This worksheet is due March 1, 2025 at midnight, to be submitted as a GitHub repository URL to Canvas. The repository should contain all files requires to compile this worksheet with your answers. You should only change this \texttt{document.tex} file and the  \texttt{references.bib} file; do not change any other file in this starting repository. You should not use any additional packages, and are not allowed to use the \texttt{{\textbackslash}usepackage\{\}} command. Additionally, the output should be formatted correctly: your answers should be appropriately nested under the questions, command-line commands should be in monospace, and images should be positioned appropriately.

\section{Git Questions}

\subsection{General questions}

\begin{enumerate}
    \item What is a version control system? Why are they useful?
        \begin{quote}
            A version control system (VCS) stores files and their history.
            It is useful to prevent data loss and see the history of a project.
        \end{quote}
    \item What is the difference between git and GitHub?
        \begin{quote}
            git is a VCS, while GitHub hosts repositories stored in git.
        \end{quote}
    \item What is a repository?
        \begin{quote}
            A repository is a collection of files stored in a VCS.
        \end{quote}
    \item What is a commit?
        \begin{quote}
            A commit describes a change to the repository.
        \end{quote}
    \item What is the commit graph?
        \begin{quote}
            The commit graph is a way to show all the commits in a repository.
        \end{quote}
    \item What is your preferred local git client (eg., command line, GitHub Desktop, GitKraken, etc.)?
        \begin{quote}
            git command line
        \end{quote}
\end{enumerate}

\subsection{Local Usage}

\begin{enumerate}
\item What is the difference between adding a file to the staging area and committing a file?
    \begin{quote}
        Adding a file to the staging area sets it up to be committed, while
        committing a file makes a commit with the file.
    \end{quote}
\item What is a commit message, and why is it important for them to be meaningful?
    \begin{quote}
        A commit message is a message to describe the commit. They are important
        to identify what the commit does and why it exists, as well as easily
        identifying past commits.
    \end{quote}
\item Starting with an empty repository, what sequence of commands/actions would result in the following commit graph? You may give a sequence of \texttt{git} commands, or describe (with screenshots) how you would do this in your preferred graphical git interface.
\begin{verbatim}
A---B---C---D
\end{verbatim}
\begin{quote}
    \begin{verbatim}
    # Add commands excluded
    git commit -m "A"
    git commit -m "B"
    git commit -m "C"
    git commit -m "D"
    \end{verbatim}
\end{quote}
\item If you are currently at commit D above, how would you recover code from commit B? What sequence of commands/actions would let you do so? You may give a sequence of \texttt{git} command-line commands, or describe (with screenshots) how you would do this in your preferred graphical git interface. Assume the commit hashes are AAAAAA..., BBBBBB..., etc.
\begin{quote}
    \begin{verbatim}
    git checkout BBBBBB
    \end{verbatim}
\end{quote}
\item Imagine you created a git repository for your project, but only commit your changes once a week on Sundays. You got your code working on Tuesday, but then broke your code on Friday. What can you do to get the working version of your code back?
    \begin{quote}
        Since the working version of the code was not committed,
        the only two options are to try to change the files back from the Friday
        version, or revert the commit to last Sunday and recreate all the
        progress up to Tuesday. This is why I commit my code pretty much
        whenever I change it.
    \end{quote}
\end{enumerate}

\subsection{Branching and Merging}

\begin{enumerate}
\item What is a branch? Why are they useful?
    \begin{quote}
        A branch is a timeline of commits.
        They are useful for working on features concurrently.
    \end{quote}
\item Starting with an empty repository, what sequence of commands/actions would result in the following commit graph? You may give a sequence of \texttt{git} command-line commands, or describe (with screenshots) how you would do this in your preferred graphical git interface.
\begin{verbatim}
A---B---C---D
     \
      E---F
\end{verbatim}
\begin{quote}
    \begin{verbatim}
    # Add commands excluded
    git commit -m "A"
    git commit -m "B"
    git commit -m "C"
    git commit -m "D"
    git checkout -b b2 BBBBBB
    git commit -m "E"
    git commit -m "F"
    \end{verbatim}
\end{quote}
\item Why is a merge? Why are they useful?
\begin{quote}
    A merge combines the changes of one branch into another.
    They are useful for combining changes you made on separate branches.
\end{quote}
\item Imagine you are currently at commit D above. What sequence of commands/actions would result in the following commit graph? You may give a sequence of \texttt{git} commands, or describe (with screenshots) how you would do this in your preferred graphical git interface.
\begin{verbatim}
A---B---C---D---G
     \         /
      E---F---/
\end{verbatim}
\begin{quote}
    \begin{verbatim}
    git merge b2 -m "G"
    \end{verbatim}
\end{quote}
\item What is a merge conflict? When do they occur?
\begin{quote}
    A merge conflict is when git can't figure out how to merge two branches
    together. They occur when both branches modify the same part of a file.
\end{quote}
\item Starting with an empty repository, despite a sequence of commands/actions that would result in a merge conflict. Include the exact edits and \texttt{git} commands or screenshots of the graphical git interface. Include the output or screenshot that shows the resulting merge conflict.
    \begin{quote}
        Start with file a. contains: a
        \begin{verbatim}
        git commit -m "a"
        \end{verbatim}
        Change contents of file a to: b
        \begin{verbatim}
        git commit -m "b"
        git checkout -b b2 AAAAAA
        \end{verbatim}
        Change contents of file a to: c
        \begin{verbatim}
        git commit -m "c"
        git checkout main
        git merge b2
        \end{verbatim}
        Output (lines broken up to fit in PDF):
        \begin{verbatim}
        Auto-merging a

        CONFLICT (content):
        Merge conflict in a

        Automatic merge failed;
        fix conflicts and then
        commit the result.
        \end{verbatim}
    \end{quote}
\end{enumerate}

\subsection{Remotes}

\begin{enumerate}
\item What is a remote?
    \begin{quote}
        A remote is an external server that hosts the repository, where changes
        can be pushed to and pulled from.
    \end{quote}
\item What does pushing and pulling do?
    \begin{quote}
        Pushing makes the remote contain any new commits
        you made on your local copy. Pulling takes any new commits from the
        remote and puts them onto your local copy.
    \end{quote}
\item Imagine you created a git repository for your project on your laptop and commit regularly, but only push your code to GitHub once a week on Sundays. Your laptop caught on fire on Friday. What can you do to get your code back?
    \begin{quote}
        It depends. Did the fire burn the entire laptop, or is the storage still
        recoverable? Assuming the storage is not recoverable, the newest version
        of the code you can get is from Sunday. This is why you should make
        regular backups of your files, as there may be more important things on
        that laptop that you could have lost. Admittedly I am really bad about
        that. Also, pushing whenever there are changes that others may want to
        see is good.
    \end{quote}
\end{enumerate}

\section{\LaTeX}

Find a source of each of the following types and add it to \texttt{references.bib}, with the appropriate data. Your sources do not have to relate to your project. Looking at \textcite{OverleafBibliographyManagement} and \textcite{WikipediaBibtex} may be helpful,

\begin{itemize}
\item a journal article
\item a conference article
\item a PhD or Master's thesis
\item an article in an edited popular media venue (newspaper, magazine, etc.)
\item a book
\item a chapter of a book
\item a YouTube video
\item a piece of technical documentation (e.g., a programming language reference, and API documentation, etc.)
\end{itemize}

Additionally, in you own words, explain the difference between \texttt{{\textbackslash}cite\{\}} and \texttt{{\textbackslash}textcite\{\}}. When should they each be used? Demonstrate your answers by using one of them with each of your references from above.

\printbibliography

\end{document}
